\documentclass{beamer}


% ---
% PACOTES
% ---

% ---
% Pacotes fundamentais 
% ---
\usepackage{times}         % Usa a fonte Latin Modern
\usepackage[T1]{fontenc}      % Selecao de codigos de fonte.
\usepackage[utf8]{inputenc}      % Codificacao do documento (conversão automática dos acentos)
\usepackage{indentfirst}      % Indenta o primeiro parágrafo de cada seção.
\usepackage{nomencl}          % Lista de simbolos
\usepackage{color}            % Controle das cores
\usepackage{graphicx}         % Inclusão de gráficos
\usepackage{microtype}        % para melhorias de justificação
% ---
% ---
% Pacotes adicionais, usados apenas no âmbito do Modelo Canônico do abnteX2
% ---
\usepackage{lipsum}           % para geração de dummy text
% ---
      
% ---
% Pacotes de citações
% ---
\usepackage[brazilian,hyperpageref]{backref}  % Paginas com as citações na bibl
\usepackage[alf]{abntex2cite} % Citações padrão ABNT
% ---

% ---
% Configurações do pacote backref
% Usado sem a opção hyperpageref de backref
\renewcommand{\backrefpagesname}{Citado na(s) página(s):~}
% Texto padrão antes do número das páginas
\renewcommand{\backref}{}
% Define os textos da citação
\renewcommand*{\backrefalt}[4]{
   \ifcase #1 %
      Nenhuma citação no texto.%
   \or
      Citado na página #2.%
   \else
      Citado #1 vezes nas páginas #2.%
   \fi}%
% ---
\usetheme{Berlin}

% --- Informações de dados para CAPA e FOLHA DE ROSTO ---
\title{Seminário - Interdisciplinaridade:
Necessidade das Ciências
Modernas e Imperativo das
Questões Ambientais}

\author{João Henrique da Silva}
\institute{UEM DCI PROFCIAMB}
\logo{
\includegraphics[width=1cm]{logo.png}
\includegraphics[width=1cm]{uem.png}
}

%\local{Brasil}
\date{Junho - 2022}
% ---

% ---
% Configurações de aparência do PDF final

% alterando o aspecto da cor azul
\definecolor{blue}{RGB}{41,5,195}

% informações do PDF
\makeatletter
\hypersetup{
      %pagebackref=true,
      pdftitle={\@title}, 
      pdfauthor={\@author},
      pdfsubject={},
      pdfcreator={},
      pdfkeywords={abnt}{latex}{abntex}{abntex2}{atigo científico}, 
      colorlinks=true,           % false: boxed links; true: colored links
      linkcolor=blue,            % color of internal links
      citecolor=blue,            % color of links to bibliography
      filecolor=magenta,            % color of file links
      urlcolor=blue,
      bookmarksdepth=4
}
\makeatother
% --- 

% ---
% compila o indice
% ---
\makeindex
% ---



% --- 
% Espaçamentos entre linhas e parágrafos 
% --- 

% O tamanho do parágrafo é dado por:
\setlength{\parindent}{1.3cm}

% Controle do espaçamento entre um parágrafo e outro:
\setlength{\parskip}{0.2cm}  % tente também \onelineskip

% Espaçamento simples
\linespread{1.3}


%\usetheme{lucid}
\begin{document}
    \frame {
        \titlepage
    }
    \frame {
        \frametitle{Ontologia e Interdisciplinaridade}
        \framesubtitle{ASPECTOS CONCEITUAIS DA INTERDISCIPLINARIDADE}
    \begin{itemize}
        \item Objetos complexos \cite[cap.6]{Paradigma_Transdisciplinar_Metodologica}
        \item Superação do reducionismo Positivista
        \item Necessidade colaboração Institucional
    \end{itemize}

    }
    \frame {
        \frametitle{Ontologia e Interdisciplinaridade}
        \framesubtitle{Gerenciamento da produção Científica }
    \begin{itemize}
        \item 1 O desenvolvimento da ciência como o resultado de dois movimentos
        \item 2 Solicitações por parte dos estudantes
        \item 3 Problemas operacionais, e administrativos, das universidades
        \item 4 Necessidade de treinamento
        \item 5 As demandas sociais
    \end{itemize}
    }
    \frame {
        \frametitle{Ontologia e Interdisciplinaridade}
        \framesubtitle{OS DESAFIOS PARA A APLICAÇÃO DOS PRINCÍPIOS DA
INTERDISCIPLINARIDADE }
    \begin{itemize}
        \item Ciência normal e paradigmas \cite{Kuhn2012-oa}
        \item Massa crítica de Cientistas - STEM
        \item Contribuições abertas
        \item Melhoria da Comunicação entre instituições
        \item Independência da comunidade
científica e tecnológica para conduzir suas pesquisas e publicá-las
    \end{itemize}
    } 
    \frame {
        \frametitle{Ontologia e Interdisciplinaridade}
        \framesubtitle{OS DESAFIOS PARA A APLICAÇÃO DOS PRINCÍPIOS DA
INTERDISCIPLINARIDADE }
    \begin{itemize}
        \item 1. A aceitação de uma metodologia comum
        \item 2. Tentativa de formulação interdisciplinar da questão maior,
        \item 3. Tradução da questão global na linguagem específica de cada disciplina
        \item 4. Verificação constante da capacidade de resposta das questões
        \item 5. Acordo sobre a resposta
científica e tecnológica para conduzir suas pesquisas e publicá-las
    \end{itemize}
    } 
    \frame {
        \frametitle{Ontologia e Interdisciplinaridade}
        \framesubtitle{PRIORIDADES ATUAIS PARA ATUAÇÃO INTERDISCIPLINAR}
    \begin{itemize}
        \item GESTÃO INTEGRADA DE BACIAS HIDROGRÁFICAS
        \item GESTÃO DA BIODIVERSIDADE
        \item MUDANÇAS CLIMÁTICAS
        \item ZONEAMENTO ECOLÓGICO E ECONÔMICO
        \item INDICADORES AMBIENTAIS
    \end{itemize}
    } 
    \frame {
        \frametitle{Ontologia e Interdisciplinaridade}
        \framesubtitle{PRIORIDADES ATUAIS PARA ATUAÇÃO INTERDISCIPLINAR}
    \begin{itemize}
        \item Condições sócioeconômicas das comunidades locais e nacionais
        \item GESTÃO DA BIODIVERSIDADE
        \item MUDANÇAS CLIMÁTICAS
        \item ZONEAMENTO ECOLÓGICO E ECONÔMICO
        \item INDICADORES AMBIENTAIS
    \end{itemize}
    } 
    \frame {
        \frametitle{Ontologia e Interdisciplinaridade}
        \framesubtitle{ZONEAMENTO ECOLÓGICO E ECONÔMICO}
    \begin{itemize}
        \item INDICADORES AMBIENTAIS
        \item REVISÃO DOS PROCESSOS DA AVALIAÇÃO DE IMPACTOS A MBIENTAIS
        \item ANÁLISE DE CICLO DE VIDA DOS PRODUTOS
        \item Blackrock group e os ESG
        \item Ecossocialismo e outras determinações econômicas
    \end{itemize}
    } 
    \frame {
        \frametitle{Ontologia e Interdisciplinaridade}
        \framesubtitle{PERSPECTIVAS FUTURAS}
    \begin{itemize}
        \item Organização da cooperação técnico-científica
        \item Consolidação de bases de dados e compartilhamento das informações
        \item Cooperação universidade-órgãos ambientais e universidade-indústria
        \item Blackrock group e os ESG
        \item Ecossocialismo e outras determinações econômicas
    \end{itemize}
    }

\bibliography{interdisciplinaridade}

\end{document}