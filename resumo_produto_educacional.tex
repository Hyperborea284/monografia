\documentclass[
   % -- opções da classe memoir --
   article,       % indica que é um artigo acadêmico
   12pt,          % tamanho da fonte
   oneside,       % para impressão apenas no recto. Oposto a twoside
   a4paper,       % tamanho do papel. 
   % -- opções da classe abntex2 --
   %chapter=TITLE,      % títulos de capítulos convertidos em letras maiúsculas
   %section=TITLE,      % títulos de seções convertidos em letras maiúsculas
   %subsection=TITLE,   % títulos de subseções convertidos em letras maiúsculas
   %subsubsection=TITLE % títulos de subsubseções convertidos em letras maiúsculas
   % -- opções do pacote babel --
   english,       % idioma adicional para hifenização
   brazil,           % o último idioma é o principal do documento
   sumario=tradicional
   ]{abntex2}


% ---
% PACOTES
% ---

% ---
% Pacotes fundamentais 
% ---
\usepackage{times}         % Usa a fonte Latin Modern
\usepackage[T1]{fontenc}      % Selecao de codigos de fonte.
\usepackage[utf8]{inputenc}      % Codificacao do documento (conversão automática dos acentos)
\usepackage{indentfirst}      % Indenta o primeiro parágrafo de cada seção.
\usepackage{nomencl}          % Lista de simbolos
\usepackage{color}            % Controle das cores
\usepackage{graphicx}         % Inclusão de gráficos
\usepackage{microtype}        % para melhorias de justificação
% ---
      
% ---
% Pacotes adicionais, usados apenas no âmbito do Modelo Canônico do abnteX2
% ---
\usepackage{lipsum}           % para geração de dummy text
% ---
      
% ---
% Pacotes de citações
% ---
\usepackage[brazilian,hyperpageref]{backref}  % Paginas com as citações na bibl
\usepackage[alf]{abntex2cite} % Citações padrão ABNT
% ---

% ---
% Configurações do pacote backref
% Usado sem a opção hyperpageref de backref
\renewcommand{\backrefpagesname}{Citado na(s) página(s):~}
% Texto padrão antes do número das páginas
\renewcommand{\backref}{}
% Define os textos da citação
\renewcommand*{\backrefalt}[4]{
   \ifcase #1 %
      Nenhuma citação no texto.%
   \or
      Citado na página #2.%
   \else
      Citado #1 vezes nas páginas #2.%
   \fi}%
% ---

% --- Informações de dados para CAPA e FOLHA DE ROSTO ---
\titulo{Resumo - Vídeo Projeto educacional}
\tituloestrangeiro{}

\autor{João Henrique da Silva}

\local{Maringá}
\data{Julho - 2022}
% ---

% ---
% Configurações de aparência do PDF final

% alterando o aspecto da cor azul
\definecolor{blue}{RGB}{41,5,195}

% informações do PDF
\makeatletter
\hypersetup{
      %pagebackref=true,
      pdftitle={\@title}, 
      pdfauthor={\@author},
      pdfsubject={},
      pdfcreator={},
      pdfkeywords={abnt}{latex}{abntex}{abntex2}{atigo científico}, 
      colorlinks=true,           % false: boxed links; true: colored links
      linkcolor=blue,            % color of internal links
      citecolor=blue,            % color of links to bibliography
      filecolor=magenta,            % color of file links
      urlcolor=blue,
      bookmarksdepth=4
}
\makeatother
% --- 

% ---
% compila o indice
% ---
\makeindex
% ---

% ---
% Altera as margens padrões
% ---
\setlrmarginsandblock{2cm}{2cm}{*}
\setulmarginsandblock{2cm}{2cm}{*}
\checkandfixthelayout
% ---

% --- 
% Espaçamentos entre linhas e parágrafos 
% --- 

% O tamanho do parágrafo é dado por:
\setlength{\parindent}{1.3cm}

% Controle do espaçamento entre um parágrafo e outro:
\setlength{\parskip}{0.2cm}  % tente também \onelineskip

% Espaçamento simples
\linespread{1.3}


% ----
% Início do documento
% ----
\begin{document}

% Seleciona o idioma do documento (conforme pacotes do babel)
%\selectlanguage{english}
\selectlanguage{brazil}

% Retira espaço extra obsoleto entre as frases.
\frenchspacing 


\maketitle


\newpage

\textual
\section{DOS FORMATOS}

O vídeo acerca do tema produto educacional apresentado por \citeonline{video_prd_ed} apresenta uma discussão acerca das características de tais produtos e detalha sua construção e aplicação. A tipificação dos formatos que pode ter um produto educacional é apresentada considerando sua aplicação e o contexto da referida profissão.

Acerca do formato de protótipo, é destacado que este produto educacional atende melhor os estudos de mestrado profissional feitos nas ciências duras e a contextualização onde este produto é aplicado é destacada. O formato de manual é discutido considerando sua operacionalização dentro do contexto profissional, e é destacado que, dentro do contexto profissional que é o ensino de ciências, este formato de produto educacional é uma forma viável e com aplicação mais dirigida ao docente. A forma do produto educacional que é o software também é apresentada, e sugere-se que sua aplicação enquanto ferramenta pode ser dirigida tanto ao docente quanto ao discente, sendo utilizado tanto para o monitoramento do ensino como para a organização dos conteúdos. É trazido também a relação de tais formas de produtos educacionais com a possibilidade de produção de patentes, situação mais comum nas produções científicas-profissionais das ciências duras. 

O formato de produto educacional que é o treinamento é apresentado, e destaca-se que este possui aplicação em diversas áreas atendidas pelos mestrados profissionais. Na área do ensino de ciências, tal produto se apresenta como uma alternativa viável e geralmente dirigida à preparação dos docentes. O formato de produto que é o material didático se destaca dentro das produções científico-profissionais vinculadas à educação, por sua ampla aplicação e familiaridade dos profissionais com o tema. Sobre esse formato, percebe-se também a utilidade enquanto ferramenta mediadora dos processos de ensino-aprendizagem. Uma outra forma de produto educacional que é a produção e divulgação artística pode ser destacada de vido sua especificidade e capacidade de comunicação, sendo também relevante para as atividade de ensino-aprendizagem devido à capacidade de informar através do apelo estético, trazendo assim referências sem a mediação formal das formas de produção do conhecimento científico formal.

Outros formatos de produto educacional são apresentados, sendo estes dotados de um campo mais estreito de aplicações. Nesta seara são destacados os formatos de zoneamento e plano diretor, nos quais a aplicação dos produtos da pesquisa científica podem servir na orientação de discussões socioespaciais pertinentes às atividades de gestão pública.

A aplicabilidade dos formatos de produto que são o relatório técnico, a divulgação técnica, a prestação de serviços e o desenvolvimento de técnicas e processos é também apresentada, e sua relação com as atividades da gestão privada podem ser inferidas. Visto que a utilidade destes formatos de conhecimento científico se apresentam em contextos similares ao contexto das formas do protótipo, infere-se que a aplicabilidade deste formato está condicionada à atividades industriais e comerciais,tipicamente privadas, e cujas operações podem ser refinadas usando tais conhecimentos.


\section{DOS PROCEDIMENTOS}

Os procedimentos na elaboração dos produtos educacionais são apresentados e as necessidades relacionadas às dimensões do domínio do conteúdo são destacadas. Sugere-se quatro dimensões do domínio dos conteúdos delimitadas primeiramente pelo seu caráter factual, onde é destacada a apropriação por parte do mestrando profissional, dos elementos que devem ser apresentados; um segundo desdobramento de tal delimitação é percebido na dimensão conceitual, sendo esta particularmente evidente nas atividades de ensino-aprendizagem atendidas pelo formato de mestrado profissional; a dimensão procedimental é destacada e dessa infere-se que a aplicabilidade deve ser um elemento central nos procedimentos de elaboração do produto educacional; por fim sugere-se que a dimensão atitudinal, centrada na conduta do mestrando profissional, deve ser considerada durante a referida elaboração.

Tais precauções embasam a necessidade de se transcender o conteudismo operacionalizado nas sequências colocadas pelos livros didáticos. Para tanto sugere-se que o produto educacional considere tais conteúdos, mas não se limite aos mesmos, incorporando outras formas de apresentação embasadas nas teorias de aprendizagem discutidas adiante.

Visando reforçar as especificidades de um mestrado profissional, é destacada a importância da produção técnica necessária para se embasar o produto educacional. A paridade de importância dada ao produto educacional, quando comparada à importância dada à dissertação é destacada e a associação deste com as teorias presentes na área do ensino é discutida. O contexto onde é dada tal associação varia conforme as especificidades de cada curso de mestrado profissional e o palestrante toma como referência a seguinte sequência, e ressalta que podem ocorrer diferenças nas etapas apresentadas.

Iniciando com o ingresso no curso, segue-se para a distribuição dos orientadores entre os aprovados, procedimento que se pauta pelas afinidades entre os projetos e os campos de pesquisa de cada orientador. Segue-se daí as discussões sobre o tema a ser estudado 
durante o curso, entendimento que embasa o procedimento seguinte, que seleciona qual será o embasamento teórico da dissertação e do produto educacional. Tal discussão acerca do embasamento teórico deve considerar o referencial teórico na área de ensino apropriado para a pesquisa proposta, e daí pode-se inferir quais questões problema e objetivos necessitam ser atendidos pela pesquisa. Conclui-se refinando a sinergia entre os objetivos e problemas ao se detalhar a aplicação dos conhecimentos que foram pesquisados para a elaboração da dissertação e produto educacional.

O palestrante ressalta a importância de se ler primeiramente o produto, antes de se conhecer a dissertação, afim de se avaliar a qualidade da produção, considerando-se também a perspectiva contida no relato de experiência da implementação proposta.


\section{DA APLICAÇÃO DA TEORIAS}

O imperativo da sinergia entre o planejamento das situações de aprendizagem e os referenciais teóricos escolhidos é discutido e daí infere-se a importância de uma seleção adequada de tais referenciais. O palestrante destaca três grupos de teorias da educação, sendo o primeiro grupo dotado de teorias cognitivistas da aprendizagem, dentro do qual se incluem os seguintes autores: Piaget, Vygotsky, Ausubel, Bruner e Gardner. Tais autores compartilham discussões acerca do papel da cognição na formação psíquica, fator que, consequentemente, afeta os processos de socialização através da educação. Segue-se classificando outros autores como comportamentalistas, grupo no qual figuram os autores Pavlov, Watson e Skinner. Tal grupo se destaca pelas discussões acerca do papel que os repertórios de comportamentos exercem sobre a formação psicossocial dos indivíduos, fator que afeta outras dimensões da educação.
Conclui-se classificando como humanistas os teóricos Maslow, Wallon e Rogers e percebe-se que o viés societal presente em tais teorias psicossociais é capaz de influenciar a produção das dissertações e produtos educacionais.

Partindo de tais entendimentos, destaca-se a sinergia entre tais fundamentações e os possíveis recursos didáticos presentes nos produtos educacionais. Ao se discutir tais sinergias sugere-se que os seguintes recursos podem estar presentes nos produtos educacionais, nomeadamente: atividades práticas, onde o discente ativamente aplica e exercita fragmentos do tema proposto; recursos das tecnologias estudadas, onde a aplicação real de tais fragmentos do tema são vivenciados pelos discentes; elementos de história da ciência, onde o discente toma conhecimento da historicidade presente nos elementos estudados; jogos e gameficação, onde o conteúdo é transformado em um jogo lúdico capaz de comunicar as especificidades do tema abordado, sem usar o formato convencional de aula ou apresentação; Conclui-se destacando como as formas da argumentação científica e das questões sociocientíficas podem conter elementos que contribuem para a aplicação das teorias psicossociais discutidas anteriormente.

Sugere-se também que a implementação do produto educacional e coleta de dados se dê em um momento anterior à qualificação, e que esta contribua para o procedimento de finalização da escrita dos trabalhos.



\postextual

\bibliography{Interdisciplinaridade}

\end{document}
