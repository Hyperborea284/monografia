\subsection{Poluição e armazenamento de energia}

Devido esta condição, o imperativo do  armazenamento de energia produzida na forma elétrica se apresenta como tema a ser investigado e este imperativo se destaca do temas das diversas formas de AE potencial. 

Acerca das infraestruturas de AE potencial, encontramos um paradigma vasto de tecnologias, das quais pode-se destacar as reservas naturais de carvão, petróleo, materiais físseis de origem natural como o urânio, materiais físseis de origem artificial como o plutônio, e reservas hidráulicas na forma de reservatórios de represas. Uma caraterística compartilhada por todas essas formas é a estocagem do insumo usado na geração da energia\footnote{\citeonline[Cap.~3-4]{Potencial_Energeticos_2050}}, e portanto, representam um paradigma tecnológico que aproveita do um legado de conhecimentos bem definido. Formas mais recentes e menos difundidas de geração da energia como a energia maremotriz, a energia geotérmica, a energia termossolar, e a energia gravitacional em reservatórios de água padecem, cada uma em sua medida, das mesmas dificuldades estruturais causadas pela inexistência de mecanismos de armazenagem de larga escala da energia produzida.

A escala da capacidade de armazenamento\footnote{\citeonline[p.159]{Potencial_Energeticos_2050}} presente nas tecnologias de baterias químicas com base em ácido e chumbo e das baterias de lítio ainda não alcançaram a grandeza escalar necessária para atuar como elemento balanceador das cargas necessárias pelas redes elétricas atuais. Embora tenha ocorrido a generalização destas formas de armazenamento em aplicações que exigem uma quantidade menor de energia, o desenvolvimento e integração de elementos balanceadores de carga em larga escala ainda não se realizou. Tal condição pode ser percebida na arquitetura dos sistemas de distribuição de energia contemporâneos, os quais conectam o lado gerador e o lado consumidor da energia de maneira praticamente instantânea.

A mensuração do percentual de produção energética renovável carrega consigo um viés oriundo do paradigma vigente que é o da inexistência\footnote{\citeonline{OECD_IEA_2012}}, dentro dos sistemas elétricos contemporâneos, de grandes elementos armazenadores da energia produzida. Tal condição implica na existência de redes energéticas onde geradores e consumidores atuam em tempo real; tem-se como exemplo e sintoma de tal arquitetura a função exercida pelo Operador Nacional do Sistema - ONS, no contexto do Brasil\footnote{\citeonline{Servicos_Ancilares}}.

Pode-se fazer uma distinção entre estes sistemas legados de armazenagem dos insumos necessários para a produção da energia frente os sistemas de armazenamento por baterias, uma vez que estas tecnologias estocam o produto, que é a energia ao invés de estocar seu insumo como o petróleo, água represada ou hidrogênio. Tal distinção estrutural é sintomática de um paradigma tecnológico emergente que atende à necessidades estruturais causadas pela infraestrutura de geração e distribuição da energia na forma elétrica e seus respectivos padrões de consumo. Existe então uma diversidade de abordagens propostas\footnote{\citeonline{Energetic_evaluation}} para o armazenamento energético em grande escala, as quais utilizam principalmente reservas de insumos, variações de processos eletroquímicos como os presentes em baterias de lítio, e sistemas mecânicos de AE também já foram propostos e implementados.

O elemento natural hidrogênio têm sido estudado e se apresenta como forma do estado da arte dentro das atividades de geração e AE. Sendo esta tecnologia capaz de atuar enquanto ferramenta de AE potencial. Tal insumo emergente portanto, se mostra como uma das formas do conhecimento reificado. A emergência de um paradigma tecnológico ao redor das tecnologias do hidrogênio é sintomática da maturidade científica deste setor da atividade humana, e suas práticas permitem que mensuremos algumas das responsabilidades colocadas pela Agenda 2030. 

Este paradigma tecnológico emergente possui o estudo da fusão do átomo de hidrogênio como seu elemento de maior destaque. Percebe-se também as contribuições do campo de estudos das células geradoras de eletricidade movidas a hidrogênio, as quais já se encontram em sistemas industriais e produtos comerciais, em menor escala. O tema da armazenagem e manuseio do hidrogênio, nas suas diversas formas, permeia este paradigma e traz consigo processos e padrões industriais que criaram um mercado próprio\footnote{\citeonline{hydrogen_markets}} e suas respectivas redes comerciais. Ademais sugere-se que a inexistência de grandes elementos armazenadores da energia supracitada possa, eventualmente, ser mitigada usando de arranjos de células geradoras de eletricidade movidas a hidrogênio associadas às tecnologias de armazenagem de hidrogênio na forma de hidreto metálico\footnote{\citeonline{Comparison_hydrogen_hydrates}}. A discussão acerca da origem verde ou não-verde\footnote{\citeonline{Developments_hydrogen}} do hidrogênio é outro elemento sintomático da maturidade deste paradigma científico e de seus conhecimentos reificados. Este fato se associa ao legado da infraestrutura energética presente e sua historicidade. Deve-se perceber também que a origem não-verde, i.e. hidrogênio produzido de fontes não renováveis, é majoritária no hidrogênio produzido, no presente momento. Fato este que entra em conflito com parte da proposta presente na Agenda 2030. Por fim, a densidade energética potencial\footnote{\citeonline{B815553B}}, e o ciclo de vida dos sistemas e dos materiais necessários para a implementação de macro sistemas energéticos, podem impactar as mensurações do percentual de produção energética renovável colocadas pela Agenda 2030.

