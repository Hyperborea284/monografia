\documentclass[
   % -- opções da classe memoir --
   article,       % indica que é um artigo acadêmico
   12pt,          % tamanho da fonte
   oneside,       % para impressão apenas no recto. Oposto a twoside
   a4paper,       % tamanho do papel. 
   % -- opções da classe abntex2 --
   %chapter=TITLE,      % títulos de capítulos convertidos em letras maiúsculas
   %section=TITLE,      % títulos de seções convertidos em letras maiúsculas
   %subsection=TITLE,   % títulos de subseções convertidos em letras maiúsculas
   %subsubsection=TITLE % títulos de subsubseções convertidos em letras maiúsculas
   % -- opções do pacote babel --
   english,       % idioma adicional para hifenização
   brazil,           % o último idioma é o principal do documento
   sumario=tradicional
   ]{abntex2}


% ---
% PACOTES
% ---

% ---
% Pacotes fundamentais 
% ---
\usepackage{times}         % Usa a fonte Latin Modern
\usepackage[T1]{fontenc}      % Selecao de codigos de fonte.
\usepackage[utf8]{inputenc}      % Codificacao do documento (conversão automática dos acentos)
\usepackage{indentfirst}      % Indenta o primeiro parágrafo de cada seção.
\usepackage{nomencl}          % Lista de simbolos
\usepackage{color}            % Controle das cores
\usepackage{graphicx}         % Inclusão de gráficos
\usepackage{microtype}        % para melhorias de justificação
% ---
      
% ---
% Pacotes adicionais, usados apenas no âmbito do Modelo Canônico do abnteX2
% ---
\usepackage{lipsum}           % para geração de dummy text
% ---
      
% ---
% Pacotes de citações
% ---
\usepackage[brazilian,hyperpageref]{backref}  % Paginas com as citações na bibl
\usepackage[alf]{abntex2cite} % Citações padrão ABNT
% ---

% ---
% Configurações do pacote backref
% Usado sem a opção hyperpageref de backref
\renewcommand{\backrefpagesname}{Citado na(s) página(s):~}
% Texto padrão antes do número das páginas
\renewcommand{\backref}{}
% Define os textos da citação
\renewcommand*{\backrefalt}[4]{
   \ifcase #1 %
      Nenhuma citação no texto.%
   \or
      Citado na página #2.%
   \else
      Citado #1 vezes nas páginas #2.%
   \fi}%
% ---

% --- Informações de dados para CAPA e FOLHA DE ROSTO ---
\titulo{A ética, a ética profissional e a educação Vol.1}
\tituloestrangeiro{Resumo do Cap.2}

\autor{João Henrique da Silva}

%\local{Brasil}
\data{Novembro - 2022}
% ---

% ---
% Configurações de aparência do PDF final

% alterando o aspecto da cor azul
\definecolor{blue}{RGB}{41,5,195}

% informações do PDF
\makeatletter
\hypersetup{
      %pagebackref=true,
      pdftitle={\@title}, 
      pdfauthor={\@author},
      pdfsubject={},
      pdfcreator={},
      pdfkeywords={abnt}{latex}{abntex}{abntex2}{atigo científico}, 
      colorlinks=true,           % false: boxed links; true: colored links
      linkcolor=blue,            % color of internal links
      citecolor=blue,            % color of links to bibliography
      filecolor=magenta,            % color of file links
      urlcolor=blue,
      bookmarksdepth=4
}
\makeatother
% --- 

% ---
% compila o indice
% ---
\makeindex
% ---

% ---
% Altera as margens padrões
% ---
\setlrmarginsandblock{2cm}{2cm}{*}
\setulmarginsandblock{2cm}{2cm}{*}
\checkandfixthelayout
% ---

% --- 
% Espaçamentos entre linhas e parágrafos 
% --- 

% O tamanho do parágrafo é dado por:
\setlength{\parindent}{1.3cm}

% Controle do espaçamento entre um parágrafo e outro:
\setlength{\parskip}{0.2cm}  % tente também \onelineskip

% Espaçamento simples
\linespread{1.3}


% ----
% Início do documento
% ----
\begin{document}

% Seleciona o idioma do documento (conforme pacotes do babel)
%\selectlanguage{english}
\selectlanguage{brazil}

% Retira espaço extra obsoleto entre as frases.
\frenchspacing 


\maketitle


\newpage

\textual
\section{Introdução}


O capítulo apresentado por \citeonline{etica_educacao_2} discute a dimensão ética do fenômeno educativo e o resultado práticos desta sobre o conhecimento humano. Para tanto o texto, dividido em quatro blocos, aborda 'a) Concepções sobre a ética; b) As ideias que adotamos sobre o que se entende pelos vocábulos “profissão e profissional”; c) Ética profissional; d) Os quatro grupos de compromissos ético pedagógicos para se forjar na prática a aliança entre a ética e a educação'\footnote{\citeonline[p.55]{etica_educacao_2}}.

A perspectiva humanista embasa a discussão acerca da ética e essa é usada para destacar o caráter societal das relações educacionais. Destas discussões é derivado o conceito de moral e este é percebido como o direcionamento necessário para a criação de uma boa vivência social. De tais argumentos infere-se que a ética pode ser percebida como uma racionalização da conduta. Espera-se que tais conhecimentos sirvam para a mediação entre a postura passiva e contemplativa e a postura pragmática necessária para a boa vivência social. A relação entre práxis ética e os exemplos dados pelas atitudes individuais fundamenta a moralidade. De tais observações deriva-se o conceito de 'ensinagem\footnote{\citeonline[p.56]{etica_educacao_2}}', necessário para destacar esta característica prática da moralidade embasada na ética. O texto entende que a racionalidade necessária para o exercício da moralidade através ética prática compartilha fundamentos racionais com a racionalidade científica. De tal observação deriva-se que a ética pode ser entendida como um elemento transdisciplinar limitador de 'verdades incautas\footnote{\citeonline[p.56]{etica_educacao_2}}' exteriores à ciência. O hábito ético calcado no pensamento científico interdisciplinar consolidou o caráter investigativo da atividade científico, e tal hábito se consolidou em detrimento de visões históricas equivocadas.

O texto apresenta a discussão presente em diversas fontes, exteriores ao escopo deste resumo, acerca da relação entre ética e moral. Para tanto, o texto apresenta as relações de causalidade entre ética e moral e destaca como a racionalidade dos regramentos vigentes no contexto societal gera tais determinações. Tais juízos de valor explicitam a lógica da moralidade vigente num determinado contexto societal. Tal racionalidade da ética é compartilhada pela práxis científica e é associada no texto com uma possível ciência do comportamento. Sugere-se também que tal racionalidade não deve empobrecer a práxis da ética através de um mecanicismo racionalista, situação que pode inclusive, mascarar o caráter societal da aplicação moral da ética. O caráter experimental da práxis ética é destacado e o texto apresenta a influência do caráter societal de tais práticas, acerca desta dimensão, sugere-se que: 

\begin{citacao}
(...) a ética social (estudo dos fenômenos éticos em pequenos ou grandes grupos sociais, tais como na família, na escola, no trabalho, na rua, no espaço público etc.), a ética política, a ética profissional, a bioética, a antropoética etc. Possui um corpo de conhecimentos próprios, que podem ser sistematizados, investigados, comprovados e cujas categorias fenomenológicas permitem ser metodizadas de forma racional e sensível \citeonline[p.58]{etica_educacao_2}. 
\end{citacao}

De tais observações infere-se que o caráter fenomenológico da ética se apresenta-se através da dimensão sensível, com destaque para a dimensão afetiva de tal fenômeno. O caráter humanista dos sentimentos despertados por tais práticas da ética, sugere o texto, pode mediar uma aproximação entre as dimensões racional-científica e moral-afetiva. 

o tema do respeito à vida é sintomático da dimensão humanista de tal problemática, e o tema da prudência é trazido afim de destacar a importância da sensibilidade na construção da racionalidade moral. É destacado também a necessidade de uma dosimetria capaz de equilibrar a relação entre as dimensões racional e afetiva, afim de se evitar a romantização desconectada da realidade. Tal caráter crítico acerca da moralidade é discutido e apresentado como sintoma da práxis ética e o caráter societal de tais atividades é apresentado. A fundamentação filosófica presente nas observações kantianas e aristotélicas é apresentada para complementar os argumentos apresentados. Sugere-se, com base em tais entendimentos, que a dimensão abstrata da atividade humana complementa o caráter empírico dos fenômenos práticos que manifestam a abstração dos entendimentos éticos e destaca-se que a mediação afetiva de tais atividades gera uma tríade biopsicoespiritual, afirma \citeonline[p.60]{etica_educacao_2} aproveitando de fontes que fogem ao escopo deste resumo.

De tais argumentos, entende-se que a ética se manifesta através do devir, onde o ser individual gera suas potencialidades sem se imiscuir de maneira passiva e reativa, sendo pautado pelo seu contexto societal. Para fundamentar tal argumentação, é destacado o papel das identidades individuais e sua relação com a alteridade societal. Sintomático de tal relação é o processo de autoconhecimento e os vícios identitários presentes na construção do self são percebidos como um viés à ser considerado. Quando se discute tal construção, percebe-se como a relação entre polarização e individualização é mediada pelas idiossincrasias e sugere-se que de tal mediação, emergem vieses cognitivos capazes de viciar a práxis moral da ética. Tais vícios se manifestam pelas diversas formas da intolerância. Sintomático de tais vieses são os julgamentos de valor universalizados pela coerção social vigente, e destaca-se que tais julgamentos convergem com o cultivo das virtudes e com o vício das pseudo virtudes, argumento que se embasa na filosofia aristotélica. Esta discussão se conecta com o imperativo da humanização freiriana, mediada pela ética no campo educacional.

O texto segue a argumentação ao incorporar as discussões sobre a dimensão profissional, e analisa a etimologia da palavra, associando esta ao ato de professar. Para tanto se destaca várias situações onde o engajamento nas diversas atividades sociais foi moldado por condições históricas e societais. Exemplo de tais condições é o termo 'métier\footnote{\citeonline[p.64]{etica_educacao_2}}', associado ao caráter liberal das profissões coligadas pelas corporações de ofício que funcionavam no contexto do surgimento do capitalismo. Tal conceito comunica a distinção entre as atividades qualificadas e não qualificadas. O texto segue a argumentação trazendo o papel das profissões no contexto da revolução industrial, contexto no qual o conceito de profissão liberal amadurece associado ao caráter elitista do termo, dado pelo contexto societal. 

O elitismo e a moralidade associados ao exercício profissional é discutido e a especificidade das atividades com base na técnica e na instrução é destacada. Tal caráter serve para a diferenciação das atividades laborais através de uma análise utilitarista das mesmas. A formalidade da instrução associada ao utilitarismo de tais atividades é apresentada como elemento de distinção destas atividades e o processo de capacitação para se exercer tais atividades gera o controle monopolístico da inserção dos indivíduos em tais atividades profissionais. O acesso ao ambiente formador dos profissionais destaca o caráter societal e político do exercício das profissões, e daí deriva-se discussões acerca da mobilidade social possível em um dado contexto histórico. Tais condições influenciam o desempenho e a qualidade das atividades profissionais, sugere o texto que as qualidades do profissional são antecedidas pelo pertencimento aos grupos sociais legitimados.

De tais condições sócio-políticas derivam situações institucionalizadas como o princípio da estabilidade e o direito à associação, projeção social e outros benefícios. O corporativismo se apresenta então como resultado de tais condicionalidades impostas pela sociedade industrial. De tais condições emerge também o fenômeno da divisão do trabalho social, gerando inclusive a diferenciação entre as atividades laborais privadas e públicas. A dimensão ética de tais atividades é trazida pelo texto ao apresentar como o 'ideal de serviço\footnote{\citeonline[p.67]{etica_educacao_2}}', elemento ideal que fundamenta parte dos elementos semânticos, técnicos e outros critérios relevantes para o pleno exercício profissional. Os determinantes econômicos e a necessidade lucro são também mediados por tais ideais que são influenciados pela ética humanista.

A análise do texto incorpora outra dimensão abstrata ao ressaltar a importância da conscientização e da 'tomada de consciência\footnote{\citeonline[p.68]{etica_educacao_2}}'. Partindo da interpretação freiriana, entende-se que o limite imposto pelos elementos cogniscíveis presentes no horizonte de significação dos indivíduos, servem de episteme para o indivíduo construir sua explicação idiossincrática da realidade. Devida daí a criticidade da conscientização individual necessária para a atuação ética na sociedade.

A dinâmica da divisão do trabalho social possui consequências políticas percebidas na disputa pelo acesso aos espaços privilegiados do poder, fato permitido pelo exercício de determinadas profissões. Ressalta o texto que a convergência de tais condicionantes materiais e abstratos moldam a ética profissional. As determinações locais e globais se apresentam então como expansão de tais potencialidades do trabalho, dentro do contexto moderno e pós-moderno. O texto aponta como sintoma desta faceta, a superação das fronteiras e destaca as implicações morais que condicionam e resultam de tal complexidade. Parte daí o entendimento de ser a ética profissional, um elemento provocador da construção de uma consciência social\footnote{\citeonline[p.70]{etica_educacao_2}}. Entende-se então que a ligação entre as formas da ética profissional e da ética pessoal são mediadas pela conscientização presente na formação da consciência individual, e que a relação entre direitos e deveres é um sintoma da complexidade vigente. Destaca-se que o reducionismo individualista pode viciar tais processos societais. Tal postura individualista se manifesta nas relações de recompensa, prêmios e punições acessíveis através do exercício profissional e esta pode ser percebida ao se analisar as representações criadas pelos profissionais acerca de si mesmos, e sobre como tais representações são difundidas nas sociedades. Ademais, adiciona-se à tais condições a dimensão normativa da atividade jurídica e percebe-se esta como uma dimensão institucionalizada de parâmetros para a atividade pautada pela ética profissional legitimadora.

Conclui-se analisando a razão ontológica da profissão percebendo as potencialidades condicionadas pelas dimensões abstratas supracitadas, na medida em que estas convergem com as condições societais historicamente construídas. Convergência esta que visa despertar a criticidade necessária para atuação ética permitida pelas atuações profissionais.

Após todas estas digressões, o texto avança sobre o tema específico da ética profissional exigida pela atividade da educação\footnote{\citeonline[p.76]{etica_educacao_2}}; daí destaca-se o caráter imperativo da educação e o impacto pedagogizante desta atividade sobre a sociedade. No nível operacional, as atividades escolares e universitárias administram tais responsabilidades e, portanto, são o locus onde o compromisso do profissional com a aprendizagem ocorrem. O texto destaca o caráter afetivo de tais compromissos educacionais e destaca a relação entre a apropriação dos conhecimentos pelos discentes, condição necessária para a plena expressão das potencialidades humanas autônomas.

O texto avança sobre o tema da 'ensinagem\footnote{\citeonline[p.78]{etica_educacao_2}}' e entende o termo como o ritual que estabelece a relação entre o saber e o indivíduo aprendente. Três dimensões desta relação são destacadas, sendo a primeira pautada pela crença conteudista, e esta é criticada por permitir um dogmatismo empobrecedor capaz de viciar o desenvolvimento ético. Contrapondo esta dimensão, o texto destaca a dimensão gnosiológica afim de destacar as omissões do conhecimento positivado e a completude dialética necessária pra a plena expressão da ensinagem. A valorização dos conhecimentos segregados releva as situações políticas que excluíram conhecimentos necessários e tais situações precisam ser estar presentes na ensinagem para que esta cumpra com sua função ética. A humanização dos indivíduos se completa ao se atentar para a terceira dimensão desta problemática. Sugere o texto que a seleção cautelosa dos conteúdos permite uma expressão plena das potencialidades dos discentes. 
Tais potencialidades se tornam claras para dos discentes na medida em que ocorre a devida politização\footnote{\citeonline[p.81]{etica_educacao_2}}, e o texto caracteriza esta como superação das posturas apolíticas. A perspectiva dos indivíduos oprimidos contribui então, com os elementos semânticos, para a plena expressão ética do indivíduo aprendente. Sintomático de tais condições é a perspectiva classista e a relação desta com o processo de humanização. O texto conclui sua argumentação com um apelo biográfico, destacando o legado das atividades educacionais e sua dimensão política na formação dos entendimentos e identidades individuais. Para tanto, aproveita-se do conceito de habitus e estrutura apresentados por Bordieu\footnote{\citeonline[p.82]{etica_educacao_2}}, afim de se destacar o impacto societal gerado do legado ético permitido pelo locus privilegiado da atividade educadora.



\postextual

\bibliography{Interdisciplinaridade}

\end{document}
