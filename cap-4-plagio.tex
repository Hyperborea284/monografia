\documentclass[
   % -- opções da classe memoir --
   article,       % indica que é um artigo acadêmico
   12pt,          % tamanho da fonte
   oneside,       % para impressão apenas no recto. Oposto a twoside
   a4paper,       % tamanho do papel. 
   % -- opções da classe abntex2 --
   %chapter=TITLE,      % títulos de capítulos convertidos em letras maiúsculas
   %section=TITLE,      % títulos de seções convertidos em letras maiúsculas
   %subsection=TITLE,   % títulos de subseções convertidos em letras maiúsculas
   %subsubsection=TITLE % títulos de subsubseções convertidos em letras maiúsculas
   % -- opções do pacote babel --
   english,       % idioma adicional para hifenização
   brazil,           % o último idioma é o principal do documento
   sumario=tradicional
   ]{abntex2}


% ---
% PACOTES
% ---

% ---
% Pacotes fundamentais 
% ---
\usepackage{times}         % Usa a fonte Latin Modern
\usepackage[T1]{fontenc}      % Selecao de codigos de fonte.
\usepackage[utf8]{inputenc}      % Codificacao do documento (conversão automática dos acentos)
\usepackage{indentfirst}      % Indenta o primeiro parágrafo de cada seção.
\usepackage{nomencl}          % Lista de simbolos
\usepackage{color}            % Controle das cores
\usepackage{graphicx}         % Inclusão de gráficos
\usepackage{microtype}        % para melhorias de justificação
% ---
      
% ---
% Pacotes adicionais, usados apenas no âmbito do Modelo Canônico do abnteX2
% ---
\usepackage{lipsum}           % para geração de dummy text
% ---
      
% ---
% Pacotes de citações
% ---
\usepackage[brazilian,hyperpageref]{backref}  % Paginas com as citações na bibl
\usepackage[alf]{abntex2cite} % Citações padrão ABNT
% ---

% ---
% Configurações do pacote backref
% Usado sem a opção hyperpageref de backref
\renewcommand{\backrefpagesname}{Citado na(s) página(s):~}
% Texto padrão antes do número das páginas
\renewcommand{\backref}{}
% Define os textos da citação
\renewcommand*{\backrefalt}[4]{
   \ifcase #1 %
      Nenhuma citação no texto.%
   \or
      Citado na página #2.%
   \else
      Citado #1 vezes nas páginas #2.%
   \fi}%
% ---

% --- Informações de dados para CAPA e FOLHA DE ROSTO ---
\titulo{Resenha - Cap. 4 O Plágio na pesquisa científica e suas consequências}
\tituloestrangeiro{}

\autor{João Henrique da Silva}

%\local{Brasil}
\data{Junho - 2022}
% ---

% ---
% Configurações de aparência do PDF final

% alterando o aspecto da cor azul
\definecolor{blue}{RGB}{41,5,195}

% informações do PDF
\makeatletter
\hypersetup{
      %pagebackref=true,
      pdftitle={\@title}, 
      pdfauthor={\@author},
      pdfsubject={},
      pdfcreator={},
      pdfkeywords={abnt}{latex}{abntex}{abntex2}{atigo científico}, 
      colorlinks=true,           % false: boxed links; true: colored links
      linkcolor=blue,            % color of internal links
      citecolor=blue,            % color of links to bibliography
      filecolor=magenta,            % color of file links
      urlcolor=blue,
      bookmarksdepth=4
}
\makeatother
% --- 

% ---
% compila o indice
% ---
\makeindex
% ---

% ---
% Altera as margens padrões
% ---
\setlrmarginsandblock{2cm}{2cm}{*}
\setulmarginsandblock{2cm}{2cm}{*}
\checkandfixthelayout
% ---

% --- 
% Espaçamentos entre linhas e parágrafos 
% --- 

% O tamanho do parágrafo é dado por:
\setlength{\parindent}{1.3cm}

% Controle do espaçamento entre um parágrafo e outro:
\setlength{\parskip}{0.2cm}  % tente também \onelineskip

% Espaçamento simples
\linespread{1.3}


% ----
% Início do documento
% ----
\begin{document}

% Seleciona o idioma do documento (conforme pacotes do babel)
%\selectlanguage{english}
\selectlanguage{brazil}

% Retira espaço extra obsoleto entre as frases.
\frenchspacing 


\maketitle


\newpage

\textual
\section{Resenha}

\subsection{Introdução}

O texto de \cite{cap_4_plagio} trata dos os desdobramentos metodológicos e éticos impostos pelos desafios da pesquisa acadêmica, e percebe isso na discussão do plágio. Percebe-se este como ato criminoso presente no ambiente Acadêmico e persistente, mesmo diante de suas consequências negativas que ferem a integridades da pesquisa científica.

Percebe-se também o papel das tecnologias que facilitam o processo de captura das ideias e a apropriação indevida das mesmas ao sugerir que as tecnologias de cópia de texto tem reforçado a situação. O tema da Preservação da propriedade intelectual traz consigo a discussão das implicações de âmbito legal e ético que podem ser minimizadas ao se combater o plágio nas diversas instituições onde esse pode aparecer.


\subsection{O Plágio na vida acadêmica}

O texto aponta o caso, em 2021, onde o ministro da Defesa alemão renunciou ao cargo após denúncias de plágio em sua tese de doutorado e percebe a situação no Brasil citando o caso do Ministro da Educação que, em 2020, passou por situação similar. O texto afirma que tais situações ainda são frequentes e comuns ocorrendo em monografias dissertações e teses, que tem apresentado tais problemas.

O texto empresta a concepção de plágio presente na publicação de Wachowickz e Costa, fonte que foge do escopo desata resenha, e que entende esse como o conflito ético que consiste no ato de tomar para si a autoria de obra intelectual de outrem, seja essa apropriação uma transcrição, uma reescrita ou uma reconfiguração dos termos que ocorre sem a devida referência à fonte original.

Como consequência, é percebido que no ambiente acadêmico, a ética e a integridade da pesquisa acabam violadas, e aparece no texto o fato de que os pesquisadores, ao buscarem a criação  de formas legítimas de um conhecimento original, acabam prejudicados. 

O caráter doloso do ato plageador é percebido tendo como base o argumento de Wachowickz e Costa que percebem também a possibilidade de se enquadrar estar atos como um procedimento fraudulento ou de ‘desleixo incompatível com quaisquer boas práticas’ (grifo no original).

O texto também percebe a dimensão jurídica de tais atitudes enquadrando o plagiador nos crimes de falsidade ideológica, presentes nos art.299 e art.184 do Código Penal que versam sobre crimes relativos à propriedade intelectual e direitos autorais.


\subsection{Os tipos de plágio}

No tópico 2.1, o texto conceitua os tipos de plágio classificando estes segundo suas características, sendo essas: o plágio integral, onde a cópia profunda e não alterada é apresentada; o plágio de tipo parcial, onde uma sequência menor das ideias contidas no texto original é aproveitada; plágio de tipo mosaico, onde diversas peças extraídas de maneira irregular acabam compondo uma peça que é apresentada como se fosse original; o misto, plágio de tipo menos consistente que os outros apresentados até agora e o plágio, chamado nesse texto, de “se faz de desentendido”, por se embasar na atitude do plagiador que apela para um suposto desconhecimento da atitude plagiadora. Segue apresentando o auto plágio, plágio do tipo que reaproveita conteúdos que não são originais e os apresentam como conteúdos originais, e conclui apresentando o plágio conceitual, sendo este aquele que se constitui do aproveitamento de ideias argumentos e linhas de raciocínio sem a devida referenciação.
A facilidade da Identificação do plágio também é percebida no texto, uma vez que esse tipo de prática pode ser percebido através da utilização de softwares específicos para verificação do originalidade dos textos.

A maneira correta de se aproveitar das informações já produzidas e presentes no contexto original das publicações científicas e acadêmicas aparece no texto ao se perceber como que a ABNT, Associação Brasileira das normas técnicas, coloca procedimentos para a correta referenciação das informações presentes em obras originais 



\subsection{Como detectar o plágio acadêmico}


O tópico 2.2 Versa sobre as maneiras correntes para se detectar o ato do plágio e para tanto os autores recorrem a outras fontes fora do escopo desse resumo.

A tipologia dos trabalhos plagiados pode ser constituída através das características desses, onde sugere-se que, como primeiro critério, perceba-se as  características da ‘necessária apresentação’, revelando-se assim desvios na apresentação formal e estrutural do documento; um segundo critério que pode ser usado na identificação do plágio deve ser a correta nomeação das fontes e sua identificação; a situação da entrega de um trabalho original é percebida como propícia para a prática do plágio; a situação da utilização de fontes que não foram suficientemente identificadas e por fim, nas situações que envolvem a obtenção de créditos e títulos acadêmicos, estas também se apresentam como situações onde pode ocorrer o plágio.


O texto apresenta também mecanismos digitais de detecção de plágio que estão presentes gratuitamente na internet, e reforça a impressão causada pela presença de poucas fontes numa determinada obra apresentada como original, situação esta que pode ser indicativa de plágio.


\subsection{Consequências da prática de plágio - Considerações finais}


O tópico 3 retoma o tema das consequências da prática de plágio percebendo que este vai contra as expectativas da comunidade acadêmica e anunciando que a instituição, assim como o indivíduo, acabam prejudicados por essa prática. tais situações tem repercussões nas instituições de ensino que já se encontra fragilizadas, e percebe também o ato falacioso ao redor da suposta produtividade do plagiador.




Os casos de plágio também podem implicar um penalizações financeiras além das implicações institucionais e penais já anunciadas, e os autores concluem o texto com uma sequência de exemplos de situações que envolveram implicações institucionais, jurídicas e indenizações.


\postextual

\bibliography{interdisciplinaridade}

\end{document}
