%%%%%%%%%%%%%%%%%%%%%%%%%%%%%%%%%%%%%%%%%%%%%%%%%%%%%%%%%%%%%%%%%%%%%%
% How to use writeLaTeX: 
%
% You edit the source code here on the left, and the preview on the
% right shows you the result within a few seconds.
%
% Bookmark this page and share the URL with your co-authors. They can
% edit at the same time!
%
% You can upload figures, bibliographies, custom classes and
% styles using the files menu.
%
%%%%%%%%%%%%%%%%%%%%%%%%%%%%%%%%%%%%%%%%%%%%%%%%%%%%%%%%%%%%%%%%%%%%%%

\documentclass[12pt]{article}

\usepackage{sbc-template}

\usepackage{graphicx,url}

%\usepackage[brazil]{babel}   
\usepackage[utf8]{inputenc}  

\usepackage{background}

\sloppy

\title{Representações sociais da poluição segundo os discentes do Novo Ensino Médio \\ Estado do conhecimento}

\author{Prof. Dr. Carlos Alberto de Oliveira Magalhães Júnior\inst{1}, Mestrando João Henrique da Silva\inst{1}}

\address{Departamento de Ciências - DCI -- Universidade Estadual de Maringá - UEM \\
        Maringá -- PR -- Brasil
        \email{juniormagalhaes@hotmail.com , 000jhs@gmail.com}
}

\backgroundsetup{
   scale=1,
   angle=0,
   opacity=1,
   color=black,
   contents={\begin{tikzpicture}[remember picture, overlay]
      \node at ([xshift=-2cm,yshift=-2cm] current page.north east)
            {\includegraphics[width = 3cm]{logo.png}} %
       node at ([xshift=2cm,yshift=-2cm] current page.north west)
            {\includegraphics[width = 3cm]{uem.png}}; %<- change the name of image
     \end{tikzpicture}}
}

\begin{document}

\maketitle

\begin{resumo} 
  Este meta-artigo descreve o estilo a ser usado na confecção de artigos e
  resumos de artigos para publicação nos anais das conferências organizadas
  pela SBC. É solicitada a escrita de resumo e abstract apenas para os artigos
  escritos em português. Artigos em inglês deverão apresentar apenas abstract.
  Nos dois casos, o autor deve tomar cuidado para que o resumo (e o abstract)
  não ultrapassem 10 linhas cada, sendo que ambos devem estar na primeira
  página do artigo.\\

  \textbf{Palavras-chave}: Representações Sociais. Novo Ensino Médio. Poluição. Tecnologia.
\end{resumo}


\section{Introdução}

All full papers and posters (short papers) submitted to some SBC conference,
including any supporting documents, should be written in English or in
Portuguese. The format paper should be A4 with single column, 3.5 cm for upper
margin, 2.5 cm for bottom margin and 3.0 cm for lateral margins, without
headers or footers. The main font must be Times, 12 point nominal size, with 6
points of space before each paragraph. Page numbers must be suppressed.

Full papers must respect the page limits defined by the conference.
Conferences that publish just abstracts ask for \textbf{one}-page texts.

\section{Metodologia}

The first page must display the paper title, the name and address of the
authors, the abstract in English and ``resumo'' in Portuguese (``resumos'' are
required only for papers written in Portuguese). The title must be centered
over the whole page, in 16 point boldface font and with 12 points of space
before itself. Author names must be centered in 12 point font, bold, all of
them disposed in the same line, separated by commas and with 12 points of
space after the title. Addresses must be centered in 12 point font, also with
12 points of space after the authors' names. E-mail addresses should be
written using font Courier New, 10 point nominal size, with 6 points of space
before and 6 points of space after.

The abstract and ``resumo'' (if is the case) must be in 12 point Times font,
indented 0.8cm on both sides. The word \textbf{Abstract} and \textbf{Resumo},
should be written in boldface and must precede the text.

\section{Resultados e discussões}

In some conferences, the papers are published on CD-ROM while only the

\subsection{Representações da poluição segundo discentes}
abstract is published in the printed Proceedings. In this case, authors are
invited to prepare two final versions of the paper.

\subsubsection{Representações da poluição segundo discentes do ensino fundamental}
abstract is published in the printed Proceedings. In this case, authors are
invited to prepare two final versions of the paper.

\subsubsection{Representações da poluição segundo discentes do Novo Ensino Médio}
abstract is published in the printed Proceedings. In this case, authors are
invited to prepare two final versions of the paper.

\subsection{Representações da poluição e geração de energia}
abstract is published in the printed Proceedings. In this case, authors are
invited to prepare two final versions of the paper.

\subsection{Representações da poluição e atividade industrial}
One, complete, to be
published on the CD and the other, containing only the first page, with
abstract and ``resumo'' (for papers in Portuguese).

\subsection{Representações da poluição, consumo e lixo}
One, complete, to be
published on the CD and the other, containing only the first page, with
abstract and ``resumo'' (for papers in Portuguese).

\section{Considerações finais}

Section titles must be in boldface, 13pt, flush left. There should be an extra
12 pt of space before each title. Section numbering is optional. The first
paragraph of each section should not be indented, while the first lines of
subsequent paragraphs should be indented by 1.27 cm.

\bibliographystyle{sbc}
\bibliography{sbc-template}

\end{document}
