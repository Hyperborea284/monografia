\documentclass{beamer}


% ---
% PACOTES
% ---

% ---
% Pacotes fundamentais 
% ---
\usepackage{times}         % Usa a fonte Latin Modern
\usepackage[T1]{fontenc}      % Selecao de codigos de fonte.
\usepackage[utf8]{inputenc}      % Codificacao do documento (conversão automática dos acentos)
\usepackage{indentfirst}      % Indenta o primeiro parágrafo de cada seção.
\usepackage{nomencl}          % Lista de simbolos
\usepackage{color}            % Controle das cores
\usepackage{graphicx}         % Inclusão de gráficos
\usepackage{microtype}        % para melhorias de justificação
% ---
% ---
% Pacotes adicionais, usados apenas no âmbito do Modelo Canônico do abnteX2
% ---
\usepackage{lipsum}           % para geração de dummy text
% ---
      
% ---
% Pacotes de citações
% ---
\usepackage[brazilian,hyperpageref]{backref}  % Paginas com as citações na bibl
\usepackage[alf]{abntex2cite} % Citações padrão ABNT
% ---

% ---
% Configurações do pacote backref
% Usado sem a opção hyperpageref de backref
\renewcommand{\backrefpagesname}{Citado na(s) página(s):~}
% Texto padrão antes do número das páginas
\renewcommand{\backref}{}
% Define os textos da citação
\renewcommand*{\backrefalt}[4]{
   \ifcase #1 %
      Nenhuma citação no texto.%
   \or
      Citado na página #2.%
   \else
      Citado #1 vezes nas páginas #2.%
   \fi}%
% ---
\usetheme{Berlin}

% --- Informações de dados para CAPA e FOLHA DE ROSTO ---
\title{Resenha - Cap. 6 Estudo de caso : Da ontologia e epistemologia aos procedimentos para a pesquisa}

\author{João Henrique da Silva\\
--
Prof. Dr. Carlos Alberto de Oliveira Magalhães Júnior}
\institute{UEM DCI PROFCIAMB}
\logo{
\includegraphics[width=1cm]{logo.png}
\includegraphics[width=1cm]{uem.png}
}

%\local{Brasil}
\date{Junho - 2022}
% ---

% ---
% Configurações de aparência do PDF final

% alterando o aspecto da cor azul
\definecolor{blue}{RGB}{41,5,195}

% informações do PDF
\makeatletter
\hypersetup{
      %pagebackref=true,
      pdftitle={\@title}, 
      pdfauthor={\@author},
      pdfsubject={},
      pdfcreator={},
      pdfkeywords={abnt}{latex}{abntex}{abntex2}{atigo científico}, 
      colorlinks=true,           % false: boxed links; true: colored links
      linkcolor=blue,            % color of internal links
      citecolor=blue,            % color of links to bibliography
      filecolor=magenta,            % color of file links
      urlcolor=blue,
      bookmarksdepth=4
}
\makeatother
% --- 

% ---
% compila o indice
% ---
\makeindex
% ---



% --- 
% Espaçamentos entre linhas e parágrafos 
% --- 

% O tamanho do parágrafo é dado por:
\setlength{\parindent}{1.3cm}

% Controle do espaçamento entre um parágrafo e outro:
\setlength{\parskip}{0.2cm}  % tente também \onelineskip

% Espaçamento simples
\linespread{1.3}


%\usetheme{lucid}
\begin{document}
    \frame {
        \titlepage
    }
    \frame {
        \frametitle{Conceitos e Proposta 1}
        \framesubtitle{Os conceitos norteadores do trabalho}
	\begin{itemize}
		\item Ontologia
		\item Metodologia
		\item Superação do reducionismo Positivista
        \item Necessidade de definir pesquisa qualitativa

	\end{itemize}
    }
    \frame{
        \frametitle{Conceitos e Proposta 2}
        \framesubtitle{Necessidade de definir pesquisa qualitativa}
        \begin{itemize}
            \item Método indutivo e a interpretação do investigador
            \item Concepção do método molda o objeto
            \item Exemplo : Maringá (lugar); resposta (28)
        \end{itemize}
    }
    \frame{
        \frametitle{Conceitos e Proposta 3}
        \framesubtitle{Abordagens na pesquisa}
        \begin{itemize}
            \item Métodos nas ciências qualitativas \cite[p.172]{cap_6_ontologia}
            \item O tipo de pesquisa 'Estudo de caso' e suas limitações
            \item Pluralismo conceitual dos estudos de caso
        \end{itemize}
    }
    \frame{
        \frametitle{Ontologia e Epistemologia}
        \framesubtitle{Apriorismos}
        \begin{itemize}
            \item Corrente Positivista; natureza estática
            \item Corrente Interpretativista; o observador altera o objeto observado
            \item Atores sociais constroem a realidade; convergência
        \end{itemize}
    }
    \frame{
        \frametitle{Tipos de estudos de caso 1}
        \framesubtitle{Unidade de análise}
        \begin{itemize}
            \item Intralocal ou plurilocal
            \item Episódico ou casuística
        \end{itemize}
    }
    \frame{
        \frametitle{Tipos de estudos de caso 2}
        \framesubtitle{Escopo da compreensão}
        \begin{itemize}
            \item Intrínseco; voltado aos casos particulares
            \item Instrumental; refinar teorias
            \item Coletivo; comparações generalizadas
        \end{itemize}
    }
    \frame{
        \frametitle{Tipos de estudos de caso 3}
        \framesubtitle{Natureza dos objetivos}
        \begin{itemize}
            \item Exploratório; obter evidências
            \item Descritivo; detalhamento
            \item Explanatório; busca a causalidade
        \end{itemize}
    }
    \frame{
        \frametitle{Procedimentos nos estudos de caso 1}
        \framesubtitle{Fluxo de trabalho}
        \begin{itemize}
            \item Visão geral do projeto de estudo de caso
            \item Descrição do processo de coleta de dados
            \item Questões norteadoras da pesquisa
            \item Formato do relatório final
        \end{itemize}
    }
    \frame{
        \frametitle{Procedimentos nos estudos de caso 2}
        \framesubtitle{Fluxo de trabalho}
        \begin{itemize}
            \item Idiossincrasias do investigador
            \item Contexto do investigador
            \item Elementos comuns que trancendem os casos
            \item Discrição sistemática dos dados e das inferências
        \end{itemize}
    }
    \frame{
        \frametitle{Procedimentos nos estudos de caso 3}
        \framesubtitle{Triangulação, uma 'arte'}
        \begin{itemize}
            \item Triangulação do investigador
            \item Triangulação dos dados
            \item Triangulação metodológica
            \item Triangulação e paralaxe; credibilidade das investigações qualitativas
        \end{itemize}
    }
    \frame{
        \frametitle{Procedimentos nos estudos de caso 4}
        \framesubtitle{Elaboração do relatório final, validade}
        \begin{itemize}
            \item Validade do construto
            \item Validade interna
            \item Validade externa
            \item Confiabilidade
            \item Convergência com \cite{Lincoln1985-ua}
        \end{itemize}
    }
    \frame{
        \frametitle{Interpretações}
        \begin{itemize}
            \item Alta empregabilidade e credibilidade científica
            \item Ontologia, epistemologia e idiossincrasias
            \item Estudo de caso na particularidade e nas generalizações
            \item Confiabilidade e triangulação dos dados
        \end{itemize}
    }
\bibliography{interdisciplinaridade}

\end{document}