\documentclass[
   % -- opções da classe memoir --
   article,       % indica que é um artigo acadêmico
   12pt,          % tamanho da fonte
   oneside,       % para impressão apenas no recto. Oposto a twoside
   a4paper,       % tamanho do papel. 
   % -- opções da classe abntex2 --
   %chapter=TITLE,      % títulos de capítulos convertidos em letras maiúsculas
   %section=TITLE,      % títulos de seções convertidos em letras maiúsculas
   %subsection=TITLE,   % títulos de subseções convertidos em letras maiúsculas
   %subsubsection=TITLE % títulos de subsubseções convertidos em letras maiúsculas
   % -- opções do pacote babel --
   english,       % idioma adicional para hifenização
   brazil,           % o último idioma é o principal do documento
   sumario=tradicional
   ]{abntex2}


% ---
% PACOTES
% ---

% ---
% Pacotes fundamentais 
% ---
\usepackage{times}         % Usa a fonte Latin Modern
\usepackage[T1]{fontenc}      % Selecao de codigos de fonte.
\usepackage[utf8]{inputenc}      % Codificacao do documento (conversão automática dos acentos)
\usepackage{indentfirst}      % Indenta o primeiro parágrafo de cada seção.
\usepackage{nomencl}          % Lista de simbolos
\usepackage{color}            % Controle das cores
\usepackage{graphicx}         % Inclusão de gráficos
\usepackage{microtype}        % para melhorias de 
\usepackage{siunitx}
\usepackage[table]{xcolor}
\usepackage[table]{xcolor}
\usepackage{graphicx}
\usepackage{longtable}
\usepackage{multirow}


%justificação
% ---
      
% ---
% Pacotes adicionais, usados apenas no âmbito do Modelo Canônico do abnteX2
% ---
\usepackage{lipsum}           % para geração de dummy text
% ---
      
% ---
% Pacotes de citações
% ---
\usepackage[brazilian,hyperpageref]{backref}  % Paginas com as citações na bibl
\usepackage[alf]{abntex2cite} % Citações padrão ABNT
% ---

% ---
% Configurações do pacote backref
% Usado sem a opção hyperpageref de backref
\renewcommand{\backrefpagesname}{Citado na(s) página(s):~}
% Texto padrão antes do número das páginas
\renewcommand{\backref}{}
% Define os textos da citação
\renewcommand*{\backrefalt}[4]{
   \ifcase #1 %
      Nenhuma citação no texto.%
   \or
      Citado na página #2.%
   \else
      Citado #1 vezes nas páginas #2.%
   \fi}%
% ---

% --- Informações de dados para CAPA e FOLHA DE ROSTO ---
\titulo{Atividade - Energia}
% \tituloestrangeiro{Resenha do Cap.2}

\autor{João Henrique da Silva}

%\local{Brasil}
\data{Junho - 2022}
% ---

% ---
% Configurações de aparência do PDF final

% alterando o aspecto da cor azul
\definecolor{blue}{RGB}{41,5,195}

% informações do PDF
\makeatletter
\hypersetup{
      %pagebackref=true,
      pdftitle={\@title}, 
      pdfauthor={\@author},
      pdfsubject={},
      pdfcreator={},
      pdfkeywords={abnt}{latex}{abntex}{abntex2}{atigo científico}, 
      colorlinks=true,           % false: boxed links; true: colored links
      linkcolor=blue,            % color of internal links
      citecolor=blue,            % color of links to bibliography
      filecolor=magenta,            % color of file links
      urlcolor=blue,
      bookmarksdepth=4
}
\makeatother
% --- 

% ---
% compila o indice
% ---
\makeindex
% ---

% ---
% Altera as margens padrões
% ---
\setlrmarginsandblock{2cm}{2cm}{*}
\setulmarginsandblock{2cm}{2cm}{*}
\checkandfixthelayout
% ---

% --- 
% Espaçamentos entre linhas e parágrafos 
% --- 

% O tamanho do parágrafo é dado por:
\setlength{\parindent}{1.3cm}

% Controle do espaçamento entre um parágrafo e outro:
\setlength{\parskip}{0.2cm}  % tente também \onelineskip

% Espaçamento simples
\linespread{1.3}

% ----
% Início do documento
% ----
\begin{document}

% Seleciona o idioma do documento (conforme pacotes do babel)
%\selectlanguage{english}
\selectlanguage{brazil}

% Retira espaço extra obsoleto entre as frases.
\frenchspacing 

\maketitle

\textual


\newpage

\section{Questão - 1}




\begin{tabular}{c c c}
    A & B & C \\
    D & E & F \\
    G & H & I \\
\end{tabular}


\begin{tabular}{||l|c|r|p{6cm}||}
    Left & Center & Right & Paragraph \\
    1 & 1 & 1 & Lorem ipsum dolor sit amet, consectetuer adipiscing elit. \\
    12 & 12 & 12 & Ut purus elit, vestibulum ut, placerat ac, adipiscing vitae, felis. \\
    123 & 123 & 123 & Curabitur dictum gravidamauris. \\
\end{tabular}



\begin{tabular}{||l|c|r|p{6cm}||}
    \cline{1-3}
    Left & Center & Right & Paragraph \\
    \hline \hline
    1 & 1 & 1 & Lorem ipsum dolor sit amet, consectetuer adipiscing elit. \\
    \hline
    12 & 12 & 12 & Ut purus elit, vestibulum ut, placerat ac, adipiscing vitae, felis. \\
    \hline
    123 & 123 & 123 & Curabitur dictum gravidamauris. \\ [.3cm]
    \cline{4-4}
\end{tabular}


% p for top aligned,
% m for vertically centered,
% b for bottom aligned
\begin{tabular}{|c|p{6cm}|}
    \hline
    Num & Top Aligned Paragraph\\
    \hline
    1 & Lorem ipsum dolor sit amet, consectetuer adipiscing elit. \\
    \hline
    12 & Ut purus elit, vestibulum ut, placerat ac, adipiscing vitae, felis. \\
    \hline
    123 & Curabitur dictum gravidamauris. \\
    \hline
\end{tabular}


% >{\raggedleft\arraybackslash} for right-justified,
% >{\centering\arraybackslash} for horizontally centered,
% >{\raggedright\arraybackslash} for left-justified
% (by default it's left-justified)
\begin{tabular}{|c|>{\raggedleft\arraybackslash}p{6cm}|}
    \hline
    Num & Top Aligned Paragraph\\
    \hline
    1 & Lorem ipsum dolor sit amet, consectetuer adipiscing elit. \\
    \hline
    12 & Ut purus elit, vestibulum ut, placerat ac, adipiscing vitae, felis. \\
    \hline
    123 & Curabitur dictum gravidamauris. \\
    \hline
\end{tabular}

%\usepackage{siunitx}
\sisetup{
  round-mode = places, % to round the numbers
  round-precision = 2, % to have two numbers after the decimal point
}


\begin{tabular}{|c|S|}
    \hline
    A & 12.34 \\
    B & 123.4\\
    C & 1.23456789 \\
    D & 12 \\
    \hline
\end{tabular}


\setlength{\tabcolsep}{.5cm}
\begin{tabular}{|c|@{\hspace{1cm}}c|c@{\hspace{.05cm}}|}
    A & B & C\\
    D & E & F\\
\end{tabular}


%\usepackage[table]{xcolor}
% ...
\begin{tabular}{|c|c|}
    \hline
    \cellcolor{purple!30}A & \cellcolor{pink!60}B \\
    \hline
    \cellcolor{red!40}C & \cellcolor{orange!50}D \\
    \hline
\end{tabular}

\renewcommand{\arraystretch}{1.5}
\begin{tabular}{|c|c|c|}
    A & B\\
    C & D\\\noalign{\smallskip}
    E & F\\[.5cm]
    G & H\\
\end{tabular}


%\usepackage[table]{xcolor}
% ...
\rowcolors{2}{gray!10}{gray!40}
\begin{tabular}{cc}
    A & B \\
    \hline
    C & D \\
    E & F \\
    G & H \\
    I & J \\
\end{tabular}

%\usepackage[table]{xcolor}
% ...
\begin{tabular}{cc >{\columncolor{green!20}}c}
    A & B & C \\
    D & E & F \\
    G & H & I \\
\end{tabular}

%\usepackage[table]{xcolor}
% ...
\setlength{\arrayrulewidth}{1.5pt}
\begin{tabular}{|c|c|c|}
    \hline
    A & B & C \\
    \hline
    D & E & F \\
    \hline
    G & H & I \\
    \hline
\end{tabular}


\begin{tabular}{ccc}
    \toprule
    \multicolumn{2}{c}{Multi-column} & \\
    \cmidrule{1-2}
    C1 & C2 & C3 \\
    \midrule
    A & B & C \\
    D & E & F \\
    \bottomrule
\end{tabular}


\begin{tabular}{|l|l|l|}
    \hline
    \multicolumn{2}{|c|}{Multi-Column} & Column 3 \\
    \hline
    Column 1 & Column 2 & Column 3 \\
    \hline
\end{tabular}


\begin{tabular}{|l|l|}
    \hline
    \multirow{2}{*}{Multi-Row} & Row 1 \\
    \cline{2-2}
    & Row 2 \\
    \hline
    Row 3 & Row 3 \\
    \hline
\end{tabular}

\begin{tabular}{|l|l|l|}
    \hline
    \multicolumn{2}{|c|}{\multirow{2}{*}{Multi-Row and Col}} & C1 \\
    \cline{3-3}
    \multicolumn{2}{|c|}{} & C2 \\
    \hline
    A3 & B3 & C3 \\
    \hline
\end{tabular}


% ...
A table example can be seen in Table \ref{table-example}.
\begin{table}[ht]
\centering
\begin{tabular}{|c|c|}
    \hline
    A & B\\
    \hline
    C & D\\
    \hline
\end{tabular}
\caption{Caption for the table}
\label{table-example}
\end{table}



%\usepackage{graphicx}
% ...
\begin{table}[h]
\resizebox{\textwidth}{!}{\begin{tabular}{|c|c|c|}
    \hline
    The first column in the table & The second column in the table & The third column in the table\\
    \hline
    1 & 2 & 3\\
    \hline
\end{tabular}}
\caption{Caption for the table}
\label{table-example}
\end{table}


%\usepackage{graphicx}
% ...
\begin{table}[h]
\centering
\scalebox{2}{\begin{tabular}{|c|c|c|}
    \hline
    A & B & C\\
    \hline
    1 & 2 & 3\\
    \hline
\end{tabular}}
\caption{Caption for the table}
\label{table-example}
\end{table}


\begin{table}[h]
\centering
\footnotesize{\begin{tabular}{|c|c|c|}
    \hline
    A & B & C\\
    \hline
    1 & 2 & 3\\
    \hline
\end{tabular}}
\caption{Caption for the table}
\label{table-example}
\end{table}

%\usepackage{longtable}
% ...
\begin{longtable}{|c|c|c|}
\caption{Caption for the multi-page table}\\
\hline
First Column & Second Column & Third Column \\
\hline
\endfirsthead
\caption{\textit{(Continued)} Caption for the multi-page table}\\
\hline
First Column & Second Column & Third Column \\
\hline
\endhead
\hline
\multicolumn{3}{r}{\textit{Continued on next page}} \\
\endfoot
\hline
\endlastfoot
X & X & X \\ X & X & X \\ X & X & X \\ X & X & X \\ X & X & X \\
X & X & X \\ X & X & X \\ X & X & X \\ X & X & X \\ X & X & X \\
X & X & X \\ X & X & X \\ X & X & X \\ X & X & X \\ X & X & X \\
X & X & X \\ X & X & X \\ X & X & X \\ X & X & X \\ X & X & X \\
X & X & X \\ X & X & X \\ X & X & X \\ X & X & X \\ X & X & X \\
X & X & X \\ X & X & X \\ X & X & X \\ X & X & X \\ X & X & X \\
X & X & X \\ X & X & X \\ X & X & X \\ X & X & X \\ X & X & X \\
X & X & X \\ X & X & X \\ X & X & X \\ X & X & X \\ X & X & X \\
X & X & X \\ X & X & X \\ X & X & X \\ X & X & X \\ X & X & X \\
X & X & X \\ X & X & X \\ X & X & X \\ X & X & X \\ X & X & X \\
\end{longtable}

\postextual

\newpage

\bibliography{Interdisciplinaridade}

\end{document}
